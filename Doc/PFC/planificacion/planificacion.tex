%==================================================================%
% Author : Pando Mu�oz, Manuel                                     %
%          S�nchez Barreiro, Pablo                                 %
% Version: 1.0, 30/03/2011                                         %                   %                                                                  %
% Memoria del Proyecto Fin de Carrera                              %
% Archivo ra�z para el cap�tulo de descripci�n general             %
%==================================================================%


\chapterheader{Descripci�n y Planificaci�n}{Descripci�n y Planificaci�n del Proyecto}
\label{chap:planificacion}

%==================================================================%
% TODO(Pablo) : Completa este p�rrafo introductorio de manera      %
%               adecuada                                           %
%==================================================================%

El presente cap\'itulo describe \todo{a completar}

\chaptertoc

\section{Descripci�n Funcional del Sistema}
\label{sec:planificacion:descFuncional}

\todo{Escribir un p�rrafo introductorio}

Es una aplicaci�n cliente servidor orientada al control del acceso a internet por parte de los alumnos mientras se realiza una prueba en salas con ordenadores en red local.

Una vez instalada en los equipos, permitir�, desde el servidor, habilitar o denegar el acceso a la red de los clientes. As� mismo, facilita tanto el env�o de ficheros desde el servidor a todos los clientes, como puede ser el enunciado o archivos necesarios para la realizaci�n, como la recepci�n y almacenaje de los resultados de cada cliente en el servidor. En el caso de la recepci�n de los resultados se efectuar� una comprobaci�n de integridad de los mismos a fin de garantizar la entrega de archivos correctos, si se detecta que un fichero se ha recibido da�ado se intentar� reenviar, si persiste el error se notificar� en los logs.

En el computador donde se ejecute el servidor se guardar�n ciertos logs tales como la hora a la que comienza o finaliza un examen o cuando finaliza un alumno particular, cuando se reciben los ficheros de resultado, posibles errores, etc
Al profesor se le da la opci�n de programar una hora de fin antes de comenzar o, si lo prefiere, finalizar manualmente. En las pruebas temporizadas desde el cliente se podr� consultar el tiempo restante.

Tanto en el servidor como en el cliente dispondr�n de interfaces gr�ficas sencillas para facilitar las operaciones.

\section{Metodolog�a de Desarrollo}
\label{sec:planificacion:metodologia}

\section{Requisitos de Alto Nivel del Sistema}
\label{sec:planificacion:requisitos}




