%=============================================================================%
% Author : Manuel Pando Muñoz                                                 %
% Author : Pablo Sánchez Barreiro                                             %
% Version: 1.0, 25/06/2011                                                    %
% Memoria del Proyecto Fin de Carrera                                         %
%=============================================================================%




\cdpchapter{Resumen}


El presente Proyecto de Fin de Carrera, perteneciente a la titulación de Ingeniería en Informática de la Universidad de Cantabria tiene como objetivo implementar un sistema de control de acceso a red durante el transcurso de pruebas evaluables en salas con computadores.
\newline

Si partimos de la base de que el deseo de los profesores es conseguir que sus alumnos aprendan y que la meta de las pruebas evaluables es determinar el grado de conocimiento de los alumnos en cierta materia, si el alumno puede consultar contenidos online o comunicarse con otros y la prueba no ha sido diseñada para permitir esto, se puede asegurar que los resultados no serán reales y ciertamente no ayuda al aprendizaje del estudiante.
\newline

Por tanto, tener un sistema que asegure el cumplimiento de los términos de la prueba y que además se aproveche de la gran utilidad que tienen las redes de computadores para hacer más cómodas ciertas tareas, parece bastante deseable. La creación de ese sistema es el objetivo de este proyecto.
\newline

Como resultado de la realización del proyecto se ha desarrollado un software que consta de una aplicación a ejecutar en el computador utilizado por el docente y otra a ejecutar en el computador de cada alumno que pretende realizar la prueba de modo que, desde la primera, se pueda tener control sobre la segunda.
\newline

Ambas aplicaciones han sido desarrolladas en Java, con interfaces gráficas simples para tratar de que el uso de los programas sea lo más intuitivo posible.
\newline

\paragraph{Palabras clave:}
 Cortafuegos, Control de acceso red, Distribuido, Linux, Java

\cdpchapter{Preface}

This Master Thesis aims to develop a network access control system for
enabling and disabling access to the local network, during the tests
conducted with computers in the University of Cantabria.
\newline

Teacher's ultimate goal is to promote their student's learning. The goal
of test is simply to measure somehow the skills of a student regarding a
certain subject. Thus, if the student can find on internet the solution
to certain questions he or she should solve by him o herself, or,
similarly, if the student can communicate with others students using the
network, we cannot ensure the results of the test reflect his or her
skills actually . Therefore, we cannot assess properly when the student
is truly learning.
\newline

So, having a system that ensures compliance with the terms of the test,
and can also take advantage of the network to automate and make more
comfortable certain tasks, seems quite desirable. The goal of this
project is to create such a system.
\newline


The result of this project was the development of a distributed
software, comprise of two different applications. The first one runs on
the teacher’s computer. The the second is executed on the computer of
every student doing the test. The teacher can control the student’s
access to the network using his or her application. So, the teacher can
grant access to the network to the students, for instance, for
distributing automatically some auxiliar material required for the test,
as well as deny access to the network.
\newline

Both applications have been developed using the Java programming
language, with simple graphical interfaces which aims to be as intuitive
as possible.
\newline

\paragraph{Keywords:}
 Firewall, Network access control, Distributed, Linux, Java


