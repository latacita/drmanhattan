%=============================================================================%
% Author : Manuel Pando Muñoz                                               %
% Author : Pablo Sánchez Barreiro                                             %  % Version: 2.0, 23/02/2011                                                    %
% Master Thesis: Introduction                                                 %
%=============================================================================


%%% Schema to write a paper introduction
%% Description of Purpose
	% What problem, issue or question does this research address ?
		%
	% What limitations or failings of current understanding, knowledge, method,
	% or technologies does this research resolve ?
		%
	% What is the significance of the problem issue or question ?
		%
%% Goal statement
	% What new understanding, knowledge, methods or technologies will this
	% research generate ?
		%
	% How this address the purpose of the work ?
		%
%% Approach
	% What experiments, prototypes or studies will be done to achieve the stated % goal ?
		%
	% How will achievement or contribution of the research be demonstrated or validated ?
		%

\chapterheader{Introducción}{Introducción}
\label{chap:introduction}

% Introducción al capítulo

Este documento es la memoria de Proyecto de Fin de Carrera en el que se muestra el proceso realizado para construir una aplicación de control de accesos a red durante pruebas evaluables. Tras una breve introducción al problema que se pretende resolver se describe la estructura del documento.

\chaptertoc

\section{Introducción}
\label{sec:intr:introduction}


Actualmente cuando se desea que un alumno no tenga acceso a la red durante pruebas evaluables, la solución es desconectar el router o switch del laboratorio, algo que sin duda es efectivo, pero neutraliza un recurso que utilizado correctamente puede ser muy útil, por ejemplo, el tener todos los computadores de una sala interconectados permite que el profesor pueda, con un gasto en tiempo muy reducido, enviar el enunciado de la prueba o ficheros necesarios en general a todos los alumnos. Algo parecido ocurre cuándo se trata de entregar los resultados, si utilizamos el método de desconectar físicamente la red, el profesor ha de utilizar algún tipo de dispositivo de almacenamiento para recoger, individual y secuencialmente los resultados de cada alumno, sin embargo, si se utiliza la red, un alumno decide cuándo enviar los resultados, sin importar si otro está haciendo lo mismo simultáneamente. El ahorro en tiempo es considerable, además se pueden realizar comprobaciones a los fichero enviados para garantizar que son correctos, ya que es un material bastante sensible, cosa que no es posible si se recogen por medio de dispositivos de almacenamiento.


\section{Estructura del Documento}
\label{sec:intr:organization}

A continuación se hace un breve resumen de los contenidos a tratar en capítulos posteriores del documento.

\paragraph{Capítulo 2: Descripción y Planificación del Proyecto} \ \\

Se describe el ámbito funcional del proyecto y la metodología escogida para su construcción. De acuerdo a ella se realiza una planificación y se enumeran los requisitos de alto nivel que el software a desarrollar ha de cumplir, así como las herramientas a utilizar durante el diseño y desarrollo.


\paragraph{Capítulo 3: Antecedentes} \ \\


Se explicarán brevemente conceptos básicos que utilizará la solución a crear y su utilidad cuándo esté completado el proyecto.


\paragraph{Capítulo 4: Definición Arquitectónica y Diseño Software} \ \\


Se explicarán mediante diagramas UML la arquitectura y el diseño propuestos para que la aplicación a desarrollar  sea correcta y segura en el cumplimiento de los requisitos, además de para tener una visión global de su funcionamiento.


\paragraph{Capítulo 5: Descripción de la primera iteración} \ \\


En este capítulo se describirá el proceso realizado en la primera iteración de la fase de construcción de acuerdo a lo explicado en el capítulo 2 sobre la metodología. En esta primera iteración se crea un sistema distribuido base, dónde el profesor puede enviar ficheros a sus alumnos.


\paragraph{Capítulo 6: Descripción de la segunda iteración} \ \\

Se describirá en este capítulo la iteración continuación a la explicada en el capítulo anterior que tiene como principal objetivo el de denegar el acceso a la red una vez que el profesor decide iniciar la prueba.
\newline

Veremos cómo gracias a seguir una metodología, el desarrollo se convierte en un proceso repetitivo que si es realizado correctamente disminuirá posibles fallos y por tanto, aumentará la calidad de la solución desarrollada.


\paragraph{Capítulo 7: Despliegue y Aceptación} \ \\


Se describirá el proceso de despliegue escogido, consistente en el empaquetamiento del software creado para posibilitar una cómoda distribución y de las pruebas realizadas una vez que se ha finalizado el desarrollo del sistema.


\paragraph{Capítulo 8: Conclusiones y Trabajos Futuros} \ \\

Se finalizará la memoria exponiendo conclusiones y enumerando una lista de posibles líneas de desarrollo futuras. 