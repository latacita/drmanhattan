%==================================================================%
% Author : Pando Muñoz, Manuel                                     %
%          Sánchez Barreiro, Pablo                                 %
% Version: 1.0, 02/03/2011                                         %
%                                                                  %
% Memoria del Proyecto Fin de Carrera                              %
% Archivo raíz para el capítulo de antecedentes                    %
%==================================================================%


\chapterheader{Antecedentes}{Antecedentes}
\label{chap:introduction}


El siguiente cap\'itulo describe brevemente las tecnolog\'ias sobre las que se fundamenta el presente proyecto. M\'as concretamente, se explica el funcionamiento de las redes locales, del framework Netfilter para el filtrado de paquetes, de los demonios en los sistemas Linux\cite{linux:2004} y del algoritmo MD5, cuatro aspectos fuertemente relacionados con el proyecto.
\newline

La aplicación a desarrollar está pensado que se utilice en los laboratorios de la Facultad de Ciencias de la Universidad de Cantabria, en estos laboratorios los computadores están conectado mediante una \emph{red local} y utilizan sistemas Linux. Para gestionar los permisos de acceso a la red se va a utilizar un \emph{demonio} que interactúe con \emph{Netfilter} por medio de iptables. Para detectar si hay algún problema en la transferencia de los ficheros por la red se utilizará MD5.


\chaptertoc

\section{Red local}

Una \emph{Red local} o LAN\cite{redes:2004}, siglas en inglés de Local Area Network, es un conjunto de computadoras conectadas entre sí en un \'area relativamente peque\~na, como los laboratorios de la Facultad.
\newline

Cada uno de estos equipos interconectados en la red se conoce como nodo. Estos nodos son capaces de enviar, recibir y procesar comandos con el fin de transportar datos, así como compartir informaci\'on y recursos a través de la red.
\newline

El funcionamiento de la red está estandarizado siendo el protocolo TCP/IP\cite{TCPIP} el más extendido, gracias a esto cualquier equipo que cumple el estándar podrá comunicarse con otros que también lo cumplan, sin importar fabricante o tecnología interna.
\newline



\section{Cortafuegos : Netfilter/iptables}
\label{sec:introduction:iptables}

Netfilter es un framework que permite filtrar de paquetes, traducci\'on de direcciones y puertos de red y varias funcionalidades m\'as para el manejo de paquetes. Es parte del n\'ucleo de Linux desde la versi\'on 2.4 del mismo, sustituyendo a ipchains, bastante limitado en comparaci\'on con Netfilter.
\newline


Para interactuar con Netfilter una de las aplicaciones más usadas es \emph{iptables}\cite{iptables:2004}, siendo necesarios permisos de administrador para ello.
\newline

Ejemplo de uso de iptables:

\begin{center}
    \# iptables -A INPUT -s 195.65.34.234 -j ACCEPT\\
\end{center}


El par\'ametro -A indica que se va a a\~nadir una regla, el objetivo de la mismo es aceptar todos los paquetes entrantes provenientes del host indicado\footnote{en vez de la IP podemos poner su FQDN (\emph{Fully Qualified Domain Name)} si lo deseáramos}. Del mismo modo, si lo que queremos es no aceptar las peticiones se cambiar\'ia ACCEPT por DROP y si nos queremos referir a los paquetes salientes OUTPUT por INPUT.
\newline

Para cambiar el tratamiento por defecto de un tipo de paquete se puede usar:

\begin{center}
    \# iptables -P INPUT DROP\\
\end{center}

Con esta orden conseguimos que por defecto, todos los paquetes entrantes (INPUT) sean rechazados (DROP). Se pueden variar los términos del mismo modo que el ejemplo anterior.
\newline


Si tenemos iptables con la configuración por defecto, se aceptan todos los paquetes entrantes y salientes, no hay ninguna restricción.




\section{Demonio}

Un \emph{demonio} (del inglés, \emph{daemon}\cite{SO:2006}) es un tipo de proceso que posee la siguientes características:

\begin{enumerate}
	\item Se ejecuta en segundo plano,
	\item Generalmente se inicia en tiempo de arranque,
	\item No usa los sistemas de entrada/salida est\'andar,
	\item Mantienen la información que necesitan en ficheros especiales bien identificados.
\end{enumerate}

Normalmente est\'an cargados en memoria esperando una señal para ser ejecutados, por lo que su gasto de recursos no suele ser significativo.
\newline


Un ejemplo claro de demonio es \emph{httpd}, que se ejecuta en los servidores web. El nombre viene de \emph{HTTP D}aemon y es utilizado para aceptar peticiones, una vez que las acepta, crea otros procesos se encargan de atenderlas.
\newline

En el proyecto se implementa un demonio para interactuar con Netfilter, que como hemos visto en la sección anterior, requiere privilegios de administrador para ello, por eso se ha decidido utilizar un demonio y no realizarlo desde la aplicación principal. Es más sencillo configurar un programa simple como el demonio, que acepta peticiones y actuar en consecuencia de las órdenes recibidas, para iniciarse en tiempo de arranque con privilegios de administrador, que no ser necesarios permisos especiales para ejecutar la aplicación del alumno.


\section{MD5}

MD5\cite{MD5:2007} son las siglas de Message-Digest Algorithm 5, algoritmo resumen del mensaje, que genera un número, conocido como firma, de 128 bits a partir del contenido de ese mensaje, de tal modo que, sí por cualquier razón, el contenido cambia, el número resultante de aplicar el algoritmo al nuevo contenido cambiará también.
\newline

Es utilizado, por ejemplo, para realizar firmas a los paquetes que nos descargamos de los repositorios para instalar aplicaciones en Linux, de este modo, si vemos que la firma MD5 del fichero que hemos descargado, no concuerda con el que debería, puede ser que se haya corrompido en el proceso de descarga, o que lo descargado no sea lo que aparenta.
\newline

En la aplicación es utilizado para comprobar la integridad de los archivos enviados por la red, se genera el MD5 en el computador local del fichero a transferir, se transfieren el fichero y el resultado del MD5, al recibir, se reconstruye el fichero y se genera un nuevo MD5 a partir de él, si coinciden, tenemos la garantía de que se ha transferido correctamente.
