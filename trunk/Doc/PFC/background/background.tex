%==================================================================%
% Author : Pando Muñoz, Manuel                                     %
%          Sánchez Barreiro, Pablo                                 %
% Version: 1.0, 02/03/2011                                         %                   %                                                                  %
% Memoria del Proyecto Fin de Carrera                              %
% Archivo raíz para el capítulo de antecedentes                    %
%==================================================================%


\chapterheader{Antecedentes}{Antecedentes}
\label{chap:introduction}

%==================================================================%
% TODO(Pablo) : Completa este párrafo introductorio de manera      %
%               adecuada                                           %
%==================================================================%

El presente cap\'itulo describe brevemente las tecnolog\'ias sobre las que se fundamenta el presente proyecto. M\'as concretamente, se explica el funcionamiento de ...\newline

\chaptertoc

\section{Red local}

%============================================================================%
% NOTA(Pablo): Poner un párrafo de como se ha llegado hasta aquí             %
%============================================================================%
Una Red local o LAN es un conjunto de computadoras conectadas entre sí en un \'area relativamente peque\~na.

%============================================================================%
% NOTA(Pablo): Poner ejemplo de área relativamente pequeña y explicar cómo   %
%              y en que se diferencian las LANs de las WANs.                 %
%============================================================================%

Cada uno de estos equipos interconectados en la red se conoce como nodo. Estos nodos son capaces de enviar, recibir y procesar comandos con el fin de transportar datos, así como compartir informaci\'on y recursos a través de la red.

%============================================================================%
% NOTA(Pablo): Esto queda suelto                                             %
%============================================================================%
El funcionamiento de la red está estandarizado siendo el protocolo TCP/IP el m\'as extendido.

%============================================================================%
% NOTA(Pablo): Esto suena a escribir por escribir, así que mejor lo quitamos %
%============================================================================%
% Entre las ventajas de las redes locales cabe destacar el ahorro en hardware, 
% si se desea que todos los equipos puedan imprimir no es necesario disponer de % una impresora para cada equipo, sirve con una conectada a la red, y del mismo % modo, el ahorro en la factura de internet, puesto que con una sola conexi\'on % se puede dar acceso a todos los equipos de la red.
%
% Otra ventaja, probablemente la m\'as importante desde el punto de vista 
% t\'ecnico es que al estar estandarizados como han de comunicarse los nodos en % la red permite la conexi\'on de equipos heterog\'eneos.
%============================================================================%

%==================================================================%
% TODO(Pablo) : Esto así a pelo queda muy duro, quizás haya que    %
%               explicar antes cómo es la arquitectura global de   %
%               la aplicación.                                     %
%==================================================================%

\section{Daemon}

% NOTA(Pablo): No se suele decir la palabra "informática", ya que está 
%              demasiado sobrecargada y no se sabe si son redes, 
%              programación, inteligencia artificial o simplemente saber
%              manejar el Word
% En inform\'atica 

Un \emph{demonio} (del inglés, \emph{daemon}) es un tipo de proceso que posee la siguientes características:

\begin{enumerate}
	\item Se ejecuta en segundo plano; 
	\item Generalmente se inicia en tiempo de arranque;
	\item No usa los sistemas de entrada/salida est\'andar;
	\item Mantienen la información que necesitan en ficheros especiales bien identificados.
\end{enumerate}

%=========================================================================%
% NOTA(Pablo): Esta información es demasiado técnica y poco interesante.  %
%              Mejor la borramos                                          %
%=========================================================================%
% En los sistemas Windows se conocen como servicios ya que son usados para, 
% precisamente, proporcionar un servicio al usuario.
% En los sistemas linux, cada daemon suele tener un script en la carpeta 
% /etc/init.d/ que permite iniciarlo, pararlo o consultar su estado.
%
% El nombre del ejecutable suele acabar en "d".
%
%=========================================================================%

Normalmente est\'an cargados en memoria esperando una se\~nal para ser ejecutados, por lo que su gasto de recursos no suele ser significativo.

% NOTA(Pablo): Pero consumen memoria, ¿no?

%=========================================================================%
% NOTA(Pablo): Esta información es demasiado técnica y poco interesante.  %
%              Mejor la borramos                                          %
%=========================================================================%
%
% Adem\'as suelen ser concurrentes, es decir, cuando se va a atender una 
% petici\'on, el daemon crea un hilo especifico para ejecutar las \'ordenes de 
% esa petici\'on concreta, de modo que el hilo principal puede seguir a la 
% espera de nuevas peticiones.\newline
%
%=========================================================================%

%=========================================================================%
% NOTA(Pablo): Faltaría poner un ejemplo de demonio y dejar clara su      % 
%              utilidad, sin entrar en detalles técnicos                  %
%              Habría que poner también la utilidad de los demonios para  %
%              el proyecto en cuestión                                    %
%=========================================================================%

%=========================================================================%
% NOTA(Pablo): Titular la sección con el nombre del concepto que          %
%              representan las ip tables                                  %
%=========================================================================%

\section{XXXXX : Iptables}

%=========================================================================%
% NOTA(Pablo): Esto sin ningún tipo de intoducción, así a bocajarro,      %
%              queda un poco duro                                         %
%=========================================================================%

Netfilter es un framework que permite filtrar de paquetes, traducci\'on de direcciones y puertos de red y varias funcionalidades m\'as para el manejo de paquetes. Es parte del n\'ucleo de linux desde la versi\'on 2.4 del mismo, sustituyendo a ipchains, bastante limitado en comparaci\'on con Netfilter.

%=========================================================================%
% NOTA(Pablo): Demasiado técnico y poco interesante, lo borramos          %
%=========================================================================%
% Iptables es una aplicaci\'on de l\'inea de comandos, que permite a un 
% usuario con privilegios de administrador, configurar un conjunto de 
% reglas para el filtrado de paquetes.
% 
%  Tambi\'en es parte del mismo proyecto que Netfilter, por lo que 
% generalmente se suele hablar solamente de iptables, puesto que es el 
% programa con el que interact\'ua directamente el usuario, para referirse 
% al d\'uo netfilter/iptables.
% 

Ejemplo de uso de iptables
\begin{center}
\# iptables -A INPUT -s 195.65.34.234 -j ACCEPT\\
\end{center}

El par\'ametro -A indica que se va a a\~nadir una regla, el objetivo de la mismo es aceptar todos los paquetes entrantes provenientes del host indicado\footnote{en vez de la IP podemos poner su FQDN (\emph{Fully Qualified Domain Name) sin lo desearamos}. Del mismo modo, si lo que queremos es no aceptar las peticiones se cambiar\'ia ACCEPT por DROP y si nos queremos referir a los paquetes salientes OUTPUT por INPUT.
%============================================================================%
% NOTA(Pablo): Por defecto ¿qué pasa cuando un paquete                       %
%              proviniente de un host llega y no se ha tocado iptables?      %
%              ¿Se rechaza o se acepta?                                      %
%============================================================================%

